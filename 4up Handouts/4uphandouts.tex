\documentclass[12pt,handout]{beamer}
% Note you must paste in the .tex file last generated from the .Rnw file 
% for this to work
\usetheme{CambridgeUS}
\usepackage{array}
\usepackage{pgfpages}
\pgfpagesuselayout{4 on 1}[letterpaper,landscape,border shrink=4mm]
\usepackage{Sweave}
\usepackage{dcolumn, multirow}
\usepackage{epsfig,graphicx}
\usepackage{verbatim, rotating, paralist,hyperref}
\usepackage{float}
\graphicspath{{./img/}} %additional graphics directory

\title{Policy Relevant Visualization and Analysis of LDS Data With Open Source Tools}

\author{Jared Knowles}
% - Use the \inst{?} command only if the authors have different
%   affiliation.

\institute[DPI] % (optional, but mostly needed)
{
  Policy Research Advisor\\
  Wisconsin Department of Public Instruction
}
\date[MIS] % (optional)
{January 20, 2012 / MIS Conference}


% If you have a file called "university-logo-filename.xxx", where xxx
% is a graphic format that can be processed by latex or pdflatex,
% resp., then you can add a logo as follows:

%\pgfdeclareimage[height=0.5cm]{university-logo}{OHSU_H_4C_POS_RGB}
%\logo{\pgfuseimage{university-logo}}

\usepackage{Sweave}
\usepackage{dcolumn, multirow}
\usepackage{epsfig,graphicx}
\usepackage{verbatim, rotating, paralist,hyperref}
\usepackage{float}
\graphicspath{{./img/}} %additional graphics directory




% If you wish to uncover everything in a step-wise fashion, uncomment
% the following command:

%\beamerdefaultoverlayspecification{<+->}

\AtBeginSection[]
{
   \begin{frame}
       \frametitle{Outline}
       \tableofcontents[currentsection]
   \end{frame}
}

\AtBeginSubsection[]
{
   \begin{frame}
       \frametitle{Outline}
       \tableofcontents[currentsection,currentsubsection]
   \end{frame}
}


\AtBeginDocument{ 
\DefineVerbatimEnvironment{Sinput}{Verbatim} {xleftmargin=2em,fontsize= 
\footnotesize} 
\DefineVerbatimEnvironment{Soutput}{Verbatim}{xleftmargin=2em,fontsize= 
\footnotesize} 
\DefineVerbatimEnvironment{Scode}{Verbatim}{xleftmargin=2em,fontsize= 
\footnotesize} 
}

\begin{document}



\begin{frame}
  \titlepage
\end{frame}


\begin{frame}{Outline}
  \tableofcontents
\end{frame}


\section{What is Policy Relevant Analysis?}
\label{sec:pol-rel-analysis}

\subsection{Defining Terms}

\begin{frame}
\frametitle{Defining Terms}
\Large \textbf{Policy relevant analysis is answering questions that inform policy in a timely fashion and presenting results in an accessible and engaging fashion.}
\end{frame}

\begin{frame}
\frametitle{The Big Questions}
  \begin{itemize}
  \item States and LEAs have an abundance of data, but how do 
  we extract meaning from it?
  \item Can we do data analysis fast enough to inform decisions and improve 
  outcomes?
  \item Can we produce analyses that are approachable to policy makers and the public so that they galvanize change?
  \item Can we do these things in a time of reduced staffing, decreased 
  budgets, and a shortage of time?
  \end{itemize}
\end{frame}

\begin{frame}
\frametitle{Example}
\begin{center}
\textbf{The state chief school officer asks: \\ 
``Do our state bilingual-bicultural programs provide any benefit to our students? Should we focus on ESL more or keep our BLBC programs?''}
\end{center}
\end{frame}

\begin{frame}
\frametitle{Options?}
  \begin{columns}
  \column[t]{.35\textwidth}
  Option
  \begin{itemize}
  \item Contract with university faculty
  \pause
  \item Consult existing literature
  \pause
  \item Call the REL
  \pause
  \end{itemize}
  \column[t]{.65\textwidth}
  Caveats
  \begin{itemize}
  \item This will take months. Budget proposal is due in three weeks. Costly.
  \pause
  \item No studies in our state. State legislators not impressed.
  \pause
  \item Study will take months. 
  \pause
  \end{itemize}
  \end{columns}
\end{frame}

{
\usebackgroundtemplate{\includegraphics[width=\paperwidth]{jen1}}
\begin{frame}[plain]
\frametitle{Or Ask Jen}
\end{frame}
}


\end{frame}

\subsection{The Problem}

{
\usebackgroundtemplate{\includegraphics[width=\paperwidth]{jen}}
\begin{frame}[plain]
\frametitle{Not Enough Jen's}
\end{frame}
}

\begin{frame}
\frametitle{How do we do more?}
\begin{center}
\Large \textbf{What tools exist to help us turn data into usable information that informs decisions?}
\end{center}
\end{frame}

\begin{frame}
\frametitle{Examples from One State}
  In Wisconsin we have done a few analyses that have helped us make decisions. 
  \begin{itemize}
  \item An analysis of the effectiveness of bilingual-bicultural programs
  \pause
  \item An analysis of reading performance and markers of struggling literacy
  \pause
  \item An analysis of concentration in dropouts
  \pause
  \item Data visualization to demonstrate problem scope for grants and press materials
  \end{itemize}
\end{frame}

\begin{frame}
\begin{center}
\Huge \textbf{How?}
\end{center}
\end{frame}

{
\usebackgroundtemplate{\includegraphics[width=\paperwidth]{fast}}
\begin{frame}[plain]
\frametitle{Policy Relevant Analysis is FAST}
\end{frame}
}

{
\usebackgroundtemplate{\includegraphics[width=\paperwidth]{narrow}}
\begin{frame}[plain]
\frametitle{Policy Relevant Analysis is FOCUSED}
\end{frame}
}

{
\usebackgroundtemplate{\includegraphics[width=\paperwidth]{horseshoes2}}
\begin{frame}[plain]
\frametitle{Policy Relevant Analysis is APPROXIMATE}
\end{frame}
}


\section{Extracting Meaning from Data}
\label{sec:extrct-meaning}

\subsection{Why Invest in Analyzing Data?}
\begin{frame}
\frametitle{Gathering More Data}
\begin{itemize}
  \item States and districts collect hundreds of attributes about millions of students
  \item Data is collected before children reach school age and after they have moved to a college or a career
  \item Patterns can tell us how choices in policy will actually affect the population
\end{itemize}
\end{frame}

{
\usebackgroundtemplate{\includegraphics[width=\paperwidth]{datain}}
\begin{frame}[plain]
\frametitle{Data is like ore}
\end{frame}
}

{
\usebackgroundtemplate{\includegraphics[width=\paperwidth]{silvergoldbars}}
\begin{frame}[plain]
\frametitle{Analysis concentrates its value}
\end{frame}
}

{
\usebackgroundtemplate{\includegraphics[width=\paperwidth]{policyproduct}}
\begin{frame}[plain]
\frametitle{And it can be used to produce something}
\end{frame}
}



\subsection{Why Can't We Invest in Data Analysis?}

\begin{frame}
\frametitle{Institutional Frustrations}
We just need to get our jobs done. We need to do them efficiently, but also transparently and in a reproducible manner. This is currently costly in time, money, and management resources.
\end{frame}

{
\usebackgroundtemplate{\includegraphics[width=\paperwidth]{frustration}}
\begin{frame}[plain]
\end{frame}
}

\begin{frame}
\frametitle{Institutional Frustrations}
We just need to get our jobs done. We need to do them efficiently, but also transparently and in a reproducible manner. This is currently costly in time, money, and management resources.
  \begin{itemize}
  \item Acquiring proprietary tools from vendors takes agreements, legal documents, and lag time
  \pause
  \item Sharing data with external researchers requires legal agreements, levels of management approval, and planning time to specify narrow scope
  \end {itemize}
\end{frame}

{
\usebackgroundtemplate{\includegraphics[width=\paperwidth]{paperwork}}
\begin{frame}[plain]
\end{frame}
}

\begin{frame}
\frametitle{Institutional Frustrations}
  \begin{itemize}
  \item Analyses are often done in proprietary tool sets, poorly documented, and unable to be reproduced with updated data later
  \end {itemize}
\end{frame}

{
\usebackgroundtemplate{\includegraphics[width=\paperwidth]{jail}}
\begin{frame}[plain]
\end{frame}
}

\begin{frame}
\frametitle{Analyses Don't Get Used}
Often when we do an in house analysis it does not get used or only gets used once.
  \begin{itemize}
  \item In house analysis often relies on the expertise of one or two staff who are obligated elsewhere. 
  \pause
  \item Analysis are done using ad-hoc tools distributed among expertise of individual staff with no comprehensive standard. 
  \pause
  \item The information we have is dependent on individual staff and the analysis projects they undertake and their tenure supporting these efforts.
  \pause
  \item Staff turnover threatens to change the information available to make decisions as knowledge leaves the agency, breaking continuity with previous reports
  \pause
  \end{itemize}
\end{frame}

{
\usebackgroundtemplate{\includegraphics[width=\paperwidth]{puzzleorg}}
\begin{frame}[plain]
\frametitle{Incoherence}
\end{frame}
}

{
\usebackgroundtemplate{\includegraphics[width=\paperwidth]{irrelevant}}
\begin{frame}[plain]
\frametitle{Irrelevant}
\end{frame}
}

\begin{frame}
\frametitle{AYP as an Example}
  \begin{itemize}
  \item Think about how AYP is calculated within your agency for SEAs, or how school performance reports are distributed to schools for LEAs
  \item Who does this task? Could someone else step in and replace them and produce the exact same results if necessary?
  \item How many other employees could use the same tools and recreate this work?
  \item Is this risk acceptable for this pivotal function?
  \end{itemize}
\end{frame}


\subsection{What is the solution?}

\begin{frame}
\frametitle{R As the Solution}
  \begin{columns}
  \column[t]{.5\textwidth}
  Objections to Data Analysis
  \begin{itemize}
  \item Costly
  \item Slow and Time Consuming
  \item Technical and complex
  \item Opaque and not actionable
  \end{itemize}
  \column[t]{.5\textwidth}
  The R Solution
  \begin{itemize}
  \item R is free and open source
  \item R allows reproducible and sharable analysis across researchers
  \item R can be scripted to do common tasks
  \item R is a lingua franca that standardizes common tasks
  \end{itemize}
  \end{columns}
\end{frame}

\begin{frame}
\frametitle{More Support for R}
  \begin{itemize}
  \item R is a common tool among data experts at major universities
  \item No need to go through procurement, R can be installed in any environment on any machine and used with no licensing or agreements needed
  \item R source code is very readable to increase transparency of processes
  \item R code is easily borrowed from and shared with others
  \item R is incredibly flexible and can be adapted to specific local needs
  \item R is under incredibly active development, improving greatly, and supported wildly by both professional and academic developers
  \end{itemize}
\end{frame}


\section{Introduction to R}
\label{sec:intro-r}

\subsection{What is R?}
\label{sec:what-r}

\begin{frame}
  \begin{itemize}
  \item R is an Open Source (and freely available) environment for statistical computinga nd graphics
  \item Available for Windows, Mac OS X, and Linux
  \item R is being actively developed with two major releases per year and dozens of releases of add on packages
  \item R can be extended with 'packages' that contain data, code, and documentation to add new functionality
  \end{itemize}
\end{frame}

\begin{frame}
\frametitle{Using R}
  \begin{columns}
  \column[t]{.5\textwidth}
  \begin{itemize}
  \item R can be used with an excellent Integrated Development Environment 
  \item RStudio makes many of the basic tasks in R much easier like
    \begin{enumerate}
    \item Importing data
    \item Previewing plots
    \item Version control 
    \item Collaboration
    \end{enumerate}
  \item Greatly increases ease of use
  \end{itemize}
\column[t]{.5\textwidth}
  \begin{center}
  \begin{figure}
  \includegraphics{rstudio-windows.png}
  \end{figure}
  \end{center}
\end{columns}
\end{frame}


\begin{frame}
\frametitle{Pros and Cons of R}
  \begin{columns}
  \column[t]{.5\textwidth}
  Pros of \textbf{R}
  \begin{itemize}
  \item Open source and freely available on all platforms
  \item Scripting for reproducible and transparent analyses
  \item Extensible to fit skills, needs, and cutting edge techniques
  \item Excellent graphical and output capabilities
  \end{itemize}
  \column[t]{.5\textwidth}
  Cons
  \begin{itemize}
  \item Steep learning curve and command line interface
  \item Requires specific inputs to get desired results
  \item Unforgiving of misspecification of inputs
  \item Data input can be tricky at first
  \end{itemize}
  \end{columns}
\end{frame}



\begin{frame}
\frametitle{Examples of R Figures}
  \begin{figure}
  \begin{center}
\includegraphics[width=.3\textwidth]{hex} 
\includegraphics[width=.3\textwidth]{crosstab} \\
\includegraphics[width=.3\textwidth]{condensity}  
\includegraphics[width=.3\textwidth]{heatmap}
  \end{center}
  \end{figure}
\end{frame}

\subsection{R Examples}
\label{sec: rexamp}

\begin{frame}
\frametitle{Inference Trees}
\begin{center}
\vspace{-.1in}
\includegraphics[width=.8\textwidth]{classtree}
\end{center}
\end{frame}

{
\usebackgroundtemplate{\includegraphics[width=\paperwidth]{classtree}}
\begin{frame}[plain]
\frametitle{Inference Trees}
\end{frame}
}


\begin{frame}
\frametitle{Visual Crosstabs}
\vspace{.1in}
\includegraphics[width=.85\textwidth]{crosstab2}
\end{frame}

\begin{frame}
\frametitle{Maps}
\begin{center}
\scalebox{.75}{
\includegraphics[width=.85\textwidth]{FRL2010}
}
\end{center}
\end{frame}

\begin{frame}
\frametitle{Heatmap}
\begin{center}
\scalebox{.75}{
\includegraphics[width=.85\textwidth]{heatmap}
}
\end{center}
\end{frame}

\begin{frame}
\frametitle{Conditional Density}
\begin{center}
\scalebox{.75}{
\includegraphics[width=.85\textwidth]{condensity}
}
\end{center}
\end{frame}

\begin{frame}
\frametitle{Faceted Hex}
\begin{center}
\scalebox{.75}{
\includegraphics[width=.85\textwidth]{hex}
}
\end{center}
\end{frame}


\begin{frame}
\frametitle{Polished Scatter}
\begin{center}
\scalebox{.9}{
\includegraphics[width=.85\textwidth]{occupationscatter}
}
\end{center}
\end{frame}


\begin{frame}
\frametitle{Longitudinal Data}
\vspace{-.1in}
\begin{center}
\scalebox{.85}{
\includegraphics[width=.95\textwidth]{g3belowproficientreadingscores}
}
\end{center}
\end{frame}


\begin{frame}
\frametitle{Proficiency Polygon}
\vspace{-.1in}
\begin{center}
\scalebox{.85}{
\includegraphics[width=.95\textwidth]{proficiencypolygon}
}
\end{center}
\end{frame}



\begin{frame}
\frametitle{Individual Growth Trajectories}
\vspace{-.15in}
\begin{center}
\scalebox{.9}{
\includegraphics[width=.95\textwidth]{facetstudents}
}
\end{center}
\end{frame}


\begin{frame}
\frametitle{Individual Growth Trajectories}
\vspace{.05in}
\begin{center}
\scalebox{.9}{
\includegraphics[width=.9\textwidth]{facetstudentspoly}
}
\end{center}
\end{frame}

\subsection{Getting StaRted}
\label{sec: start}

\begin{frame}[containsverbatim]
\frametitle{The Command Line}
\begin{itemize}
\item R can be tricky because it uses command lines.
\item This is powerful, but requires a learning curve.
\item Some simple calculations can give a feel for how R works
\end{itemize}
\begin{Schunk}
\begin{Sinput}
> 2+2
\end{Sinput}
\begin{Soutput}
[1] 4
\end{Soutput}
\begin{Sinput}
> 7*4
\end{Sinput}
\begin{Soutput}
[1] 28
\end{Soutput}
\begin{Sinput}
> exp(3)
\end{Sinput}
\begin{Soutput}
[1] 20.08554
\end{Soutput}
\begin{Sinput}
> pi
\end{Sinput}
\begin{Soutput}
[1] 3.141593
\end{Soutput}
\end{Schunk}
\end{frame}

\begin{frame}[containsverbatim]
\frametitle{Deconstruct R Commands}
\begin{Schunk}
\begin{Sinput}
> summary(student_long[,28:31])
\end{Sinput}
\begin{Soutput}
     readSS          mathSS             proflvl      
 Min.   :200.0   Min.   :200.0   below basic: 37618  
 1st Qu.:430.3   1st Qu.:419.5   basic      : 85322  
 Median :495.6   Median :480.9   proficient :195231  
 Mean   :495.1   Mean   :483.7   advanced   :131829  
 3rd Qu.:559.5   3rd Qu.:545.4                       
 Max.   :866.9   Max.   :839.9                       
 race      
 A:  8802  
 B:185748  
 H: 50025  
 I:  3732  
 W:201693  
\end{Soutput}
\end{Schunk}
\begin{itemize}
  \item \textbf{summary} is the function
  \item \textbf{student\_ long} is the data object
  %\item \textbf{[,27:31]} is an operator that specified conditions to operate under
\end{itemize}
\end{frame}

\begin{frame}[containsverbatim]
\frametitle{Simple R Operations}
\begin{Schunk}
\begin{Sinput}
> with(student_long,mean(readSS[year=='2001' & grade==4]))
\end{Sinput}
\begin{Soutput}
[1] 445.0533
\end{Soutput}
\end{Schunk}
\begin{Schunk}
\begin{Sinput}
> with(student_long,median(readSS[year=='2001' & grade==4]))
\end{Sinput}
\begin{Soutput}
[1] 442.5596
\end{Soutput}
\end{Schunk}
\begin{Schunk}
\begin{Sinput}
> with(student_long,max(readSS[year=='2001' & grade==4]))
\end{Sinput}
\begin{Soutput}
[1] 721.4892
\end{Soutput}
\end{Schunk}
\begin{Schunk}
\begin{Sinput}
> with(student_long,min(readSS[year=='2001' & grade==4]))
\end{Sinput}
\begin{Soutput}
[1] 200
\end{Soutput}
\end{Schunk}
\begin{Schunk}
\begin{Sinput}
> with(student_long,summary(readSS[year=='2001' & grade==4]))
\end{Sinput}
\begin{Soutput}
   Min. 1st Qu.  Median    Mean 3rd Qu.    Max. 
  200.0   393.3   442.6   445.1   494.7   721.5 
\end{Soutput}
\end{Schunk}
\end{frame}

  
\begin{frame}[containsverbatim]
\frametitle{Crosstabs}
Let's test for balance among some categories of students 
\begin{Schunk}
\begin{Sinput}
> with(subset(student_long,year=='2001' 
+             & grade==3),table(female,race))
\end{Sinput}
\begin{Soutput}
      race
female    A    B    H    I    W
     0  234 5087 1414   99 5504
     1  255 5209 1381  113 5704
\end{Soutput}
\end{Schunk}
\begin{Schunk}
\begin{Sinput}
> #As proportions
> with(subset(student_long,year=='2001' 
+             & grade==3),round(prop.table
+             (table(female,race))*100),4)
\end{Sinput}
\begin{Soutput}
      race
female  A  B  H  I  W
     0  1 20  6  0 22
     1  1 21  6  0 23
\end{Soutput}
\end{Schunk}
\end{frame}

\begin{frame}
\frametitle{Crosstabs}
We can even output the results of R commands into a print-ready format. As we have below.
% latex table generated in R 2.14.1 by xtable 1.6-0 package
% Wed Feb  1 11:36:36 2012
\begin{table}[ht]
\begin{center}
\begin{tabular}{rrrrrr}
  \hline
 & A & B & H & I & W \\ 
  \hline
0 & 1.00 & 20.00 & 6.00 & 0.00 & 22.00 \\ 
  1 & 1.00 & 21.00 & 6.00 & 0.00 & 23.00 \\ 
   \hline
\end{tabular}
\end{center}
\end{table}\end{frame}

\section{More Advanced Functions}
\subsection{Analysis}

\begin{frame}
\frametitle{Doing More than the Basics}
  \begin{itemize}
  \item R can routinize basic functions like tables, crosstabs, and visualization of data
  \item R can also be extended to do more advanced analyses like multilevel modeling, spatial error modeling, Bayesian data analysis, forecasting, and simulation
  \item R can do advanced graphical functions as well
  \item R can even be expanded to incorporate additional programming languages like Python, C++, and Java
  \end{itemize}
\textbf{The downside of this is that these functions can have a steep learning curve.}
\end{frame}

\begin{frame}[containsverbatim]
\frametitle{ANOVA}
We can also do statistical tests using both Bayesian and Frequentist methods.
\begin{Schunk}
\begin{Sinput}
> nova1<-aov(readSS~female*race*econ,data=novaset)
> summary(nova1)
\end{Sinput}
\begin{Soutput}
                    Df   Sum Sq  Mean Sq  F value   Pr(>F)
female               1  2097543  2097543  728.507  < 2e-16
race                 4 40225804 10056451 3492.748  < 2e-16
econ                 1 22330214 22330214 7755.600  < 2e-16
female:race          4    81042    20261    7.037 1.17e-05
female:econ          1     6797     6797    2.361   0.1244
race:econ            4   189430    47358   16.448 1.83e-13
female:race:econ     4    23672     5918    2.055   0.0838
Residuals        24980 71923350     2879                  
                    
female           ***
race             ***
econ             ***
female:race      ***
female:econ         
race:econ        ***
female:race:econ .  
Residuals           
---
Signif. codes:  0 ‘***’ 0.001 ‘**’ 0.01 ‘*’ 0.05 ‘.’ 0.1 ‘ ’ 1 
\end{Soutput}
\end{Schunk}
\end{frame}

\begin{frame}
\frametitle{Pretty Output}
We can also do print-ready model outputs with R's extensible formatting
% latex table generated in R 2.14.1 by xtable 1.6-0 package
% Wed Feb  1 11:36:37 2012
\begin{table}[ht]
\begin{center}
\begin{tabular}{lrrrrr}
  \hline
 & Df & Sum Sq & Mean Sq & F value & Pr($>$F) \\ 
  \hline
female           & 1 & 2097543.47 & 2097543.47 & 728.51 & 0.0000 \\ 
  race             & 4 & 40225803.97 & 10056450.99 & 3492.75 & 0.0000 \\ 
  econ             & 1 & 22330213.97 & 22330213.97 & 7755.60 & 0.0000 \\ 
  female:race      & 4 & 81042.06 & 20260.51 & 7.04 & 0.0000 \\ 
  female:econ      & 1 & 6797.30 & 6797.30 & 2.36 & 0.1244 \\ 
  race:econ        & 4 & 189430.29 & 47357.57 & 16.45 & 0.0000 \\ 
  female:race:econ & 4 & 23671.91 & 5917.98 & 2.06 & 0.0838 \\ 
  Residuals        & 24980 & 71923350.40 & 2879.24 &  &  \\ 
   \hline
\end{tabular}
\end{center}
\end{table}\end{frame}

\subsection{Linear Model Example}

\begin{frame}[containsverbatim,allowframebreaks,fragile]
\frametitle{A simple OLS Model}
\begin{Schunk}
\begin{Sinput}
> mod1<-lm(readSS~female*race*econ+grade*year,data=student_long)
> summary(mod1)
\end{Sinput}
\begin{Soutput}
Call:
lm(formula = readSS ~ female * race * econ + grade * year, data = student_long)

Residuals:
     Min       1Q   Median       3Q      Max 
-232.761  -34.179    0.123   34.270  264.081 

Coefficients:
                   Estimate Std. Error t value Pr(>|t|)    
(Intercept)        373.8444     1.4086 265.397  < 2e-16 ***
female              16.3185     1.8846   8.659  < 2e-16 ***
raceB              -39.3492     1.4405 -27.315  < 2e-16 ***
raceH              -35.1787     1.5318 -22.966  < 2e-16 ***
raceI                3.6151     2.4163   1.496   0.1346    
raceW                8.9380     1.3582   6.581 4.68e-11 ***
econ1              -77.9826     1.6455 -47.391  < 2e-16 ***
grade               22.0576     0.0776 284.258  < 2e-16 ***
year2001             2.7031     0.6320   4.277 1.89e-05 ***
year2002           140.6231     0.6320 222.506  < 2e-16 ***
female:raceB         1.9210     2.0292   0.947   0.3438    
female:raceH        -1.5938     2.1502  -0.741   0.4585    
female:raceI         4.6054     3.4470   1.336   0.1815    
female:raceW        -2.6560     1.9121  -1.389   0.1648    
female:econ1        -1.2084     2.3147  -0.522   0.6016    
raceB:econ1         13.5581     1.7386   7.798 6.28e-15 ***
raceH:econ1         28.8879     1.8427  15.677  < 2e-16 ***
raceI:econ1          2.5158     3.0038   0.838   0.4023    
raceW:econ1         14.8205     1.6773   8.836  < 2e-16 ***
grade:year2001       8.2691     0.1097  75.353  < 2e-16 ***
grade:year2002      -5.1974     0.1097 -47.361  < 2e-16 ***
female:raceB:econ1  -5.0485     2.4468  -2.063   0.0391 *  
female:raceH:econ1  -2.2068     2.5869  -0.853   0.3936    
female:raceI:econ1  -4.4296     4.2397  -1.045   0.2961    
female:raceW:econ1   0.2727     2.3594   0.116   0.9080    
---
Signif. codes:  0 ‘***’ 0.001 ‘**’ 0.01 ‘*’ 0.05 ‘.’ 0.1 ‘ ’ 1 

Residual standard error: 51.33 on 449975 degrees of freedom
Multiple R-squared: 0.6852,  Adjusted R-squared: 0.6852 
F-statistic: 4.081e+04 on 24 and 449975 DF,  p-value: < 2.2e-16 
\end{Soutput}
\end{Schunk}
\end{frame}

\begin{frame}
\frametitle{Model Evaluation}
\begin{center}
\vspace{-.1in}
\includegraphics[width=.8\textwidth]{residplot}
\end{center}
\end{frame}

\begin{frame}
\frametitle{Model Evaluation Part 2}
\vspace{-.1in}
\begin{center}
\includegraphics[width=.8\textwidth]{fittedplot}
\end{center}
\end{frame}

\begin{frame}
\frametitle{Better Fitting}
Using advanced techniques we can greatly improve our model fit over the OLS model.
\vspace{-.1in}
\begin{center}
\includegraphics[height=.7\paperheight]{mlfitted}
\end{center}
\end{frame}


\begin{frame}
\frametitle{Compare OLS and Mixed Effects}
A simple mixed-effects model estimated with an R package can outperform the simple OLS without much additional effort
\vspace{-.1in}
\begin{center}
\includegraphics[height=.6\paperheight]{OLSandMLM}
\end{center}
\end{frame}



\end{document}

% This file is a solution template for:
% - Giving a talk on some subject.
% - The talk is between 15min and 45min long.
% - Style is ornate.
% Copyright 2004 by Till Tantau <tantau@users.sourceforge.net>.
%
% In principle, this file can be redistributed and/or modified under
% the terms of the GNU Public License, version 2.
%
% However, this file is supposed to be a template to be modified
% for your own needs. For this reason, if you use this file as a
% template and not specifically distribute it as part of a another
% package/program, I grant the extra permission to freely copy and
% modify this file as you see fit and even to delete this copyright
% notice.