\documentclass{beamer}
\usetheme{CambridgeUS}
\usepackage{array}


\title [Policy Relevant Research with LDS Data]{Policy Relevant Visualization and Analysis of LDS Data With Open Source Tools}

\author{Jared Knowles}
% - Use the \inst{?} command only if the authors have different
%   affiliation.

\institute[DPI] % (optional, but mostly needed)
{
  Policy Research Advisor\\
  Wisconsin Department of Public Instruction
}
\date[MIS] % (optional)
{January 20, 2012 / MIS Conference}


% If you have a file called "university-logo-filename.xxx", where xxx
% is a graphic format that can be processed by latex or pdflatex,
% resp., then you can add a logo as follows:

%\pgfdeclareimage[height=0.5cm]{university-logo}{OHSU_H_4C_POS_RGB}
%\logo{\pgfuseimage{university-logo}}

\usepackage{Sweave}
\usepackage{dcolumn, multirow}
\usepackage{epsfig,graphicx,multicol}
\usepackage{verbatim, rotating, paralist,hyperref}
\usepackage{float,color}
%\usepackage[usenames,dvipsnames]{xcolor}
\graphicspath{{./img/}} %additional graphics directory




% If you wish to uncover everything in a step-wise fashion, uncomment
% the following command:

%\beamerdefaultoverlayspecification{<+->}

\AtBeginSection[]
{
   \begin{frame}
       \frametitle{Outline}
        \begin{multicols}{2} 
       \tableofcontents[currentsection]
        \end{multicols} 
   \end{frame}
}



\AtBeginSubsection[]
{
   \begin{frame}
       \frametitle{Outline}
       \begin{multicols}{2} 
       \tableofcontents[currentsection,currentsubsection]
       \end{multicols} 
   \end{frame}
}


\AtBeginDocument{ 
\DefineVerbatimEnvironment{Sinput}{Verbatim} {xleftmargin=2em,fontsize= 
\footnotesize,fontseries=bc,frame=single} 
\DefineVerbatimEnvironment{Soutput}{Verbatim}{xleftmargin=2em,fontsize= 
\footnotesize,frame=leftline} 
\DefineVerbatimEnvironment{Scode}{Verbatim}{xleftmargin=2em,fontsize= 
\footnotesize,fontseries=bc} 
}

\begin{document}



\begin{frame}
  \titlepage
\end{frame}


% \begin{frame}{Outline}
%   \tableofcontents
% \end{frame}


\section{What is Policy Relevant Analysis?}
\label{sec:pol-rel-analysis}

\subsection{Defining Terms}

\begin{frame}
\frametitle{Defining Terms}
\Large \textbf{Policy relevant analysis is answering questions that inform policy in a timely fashion and presenting results in an accessible and engaging fashion.}
\end{frame}

\begin{frame}
\frametitle{The Big Questions}
  \begin{itemize}
  \item States and LEAs have an abundance of data, but how do 
  we extract meaning from it?
  \item Can we do data analysis fast enough to inform decisions and improve 
  outcomes?
  \item Can we produce analyses that are approachable to policy makers and the public so that they galvanize change?
  \item Can we do these things in a time of reduced staffing, decreased 
  budgets, and a shortage of time?
  \end{itemize}
\end{frame}

\begin{frame}
\frametitle{Example}
\begin{center}
\textbf{The state chief school officer asks: \\ 
``Do our state bilingual-bicultural programs provide any benefit to our students? Should we focus on ESL more or keep our BLBC programs?''}
\end{center}
\end{frame}

\begin{frame}
\frametitle{Options?}
  \begin{columns}
  \column[t]{.35\textwidth}
  Option
  \begin{itemize}
  \item Contract with university faculty
  \pause
  \item Consult existing literature
  \pause
  \item Call the REL
  \pause
  \end{itemize}
  \column[t]{.65\textwidth}
  Caveats
  \begin{itemize}
  \item This will take months. Budget proposal is due in three weeks. Costly.
  \pause
  \item No studies in our state. State legislators not impressed.
  \pause
  \item Study will take months. 
  \pause
  \end{itemize}
  \end{columns}
\end{frame}

{
\usebackgroundtemplate{\includegraphics[width=\paperwidth]{jen1}}
\begin{frame}[plain]
\frametitle{Or Ask Jen}
\end{frame}
}


\end{frame}

\subsection{The Problem}

{
\usebackgroundtemplate{\includegraphics[width=\paperwidth]{jen}}
\begin{frame}[plain]
\frametitle{Not Enough Jen's}
\end{frame}
}

\begin{frame}
\frametitle{How do we do more?}
\begin{center}
\Large \textbf{What tools exist to help us turn data into usable information that informs decisions?}
\end{center}
\end{frame}


\begin{frame}
\frametitle{Summary}
\begin{center}
\Large \textbf{An approximate answer to the right problem is worth a good deal more than an exact answer to an approximate problem.}
\\
\end{center}
\large John Tukey
\end{frame}


\begin{frame}
\frametitle{Policy Relevant Research}
  \begin{columns}
  \column[t]{.5\textwidth}
  Policy relevant research is...
  \begin{itemize}
  \item Fast
  \pause
  \item Narrowly focused
  \pause
  \item Approximate
  \pause
  \item Fast
  \end{itemize}
  \column[t]{.5\textwidth}
  Traditional research is...
  \begin{itemize}
  \item Long term
  \pause
  \item Branches to new ideas and new paths
  \pause
  \item Focused on precision 
  \pause
  \item Peer-reviewed
  \end{itemize}
  \end{columns}
\end{frame}

\begin{frame}
\frametitle{Policy Relevant Research Is Not...}
  \begin{itemize}
  \item A visualization and summary statistics
  \item Descriptive statistics organized with some text
  \item Ignorant of policy limitations and context
  \end{itemize}
\begin{center}
But it also is not:
\end{center}
  \begin{itemize}
    \item Focused on purely causal relationships
    \item Overly concerned with precision of estimates
    \item Irrefutable 
  \end{itemize}
\end{frame}

\begin{frame}
\frametitle{Examples from One State}
  In Wisconsin we have done a few analyses that have helped us make decisions. 
  \begin{itemize}
  \item An analysis of the effectiveness of bilingual-bicultural programs
  \pause
  \item An analysis of reading performance and markers of struggling literacy
  \pause
  \item An analysis of concentration in dropouts
  \pause
  \item Data visualization to demonstrate problem scope for grants and press materials
  \end{itemize}
\end{frame}

\begin{frame}
\begin{center}
\Huge \textbf{How?}
\end{center}
\end{frame}

{
\usebackgroundtemplate{\includegraphics[width=\paperwidth]{fast}}
\begin{frame}[plain]
\frametitle{Policy Relevant Analysis is FAST}
\end{frame}
}

{
\usebackgroundtemplate{\includegraphics[width=\paperwidth]{laserfocus}}
\begin{frame}[plain]
\frametitle{Policy Relevant Analysis is FOCUSED}
\end{frame}
}

{
\usebackgroundtemplate{\includegraphics[width=\paperwidth]{horseshoes2}}
\begin{frame}[plain]
\frametitle{Policy Relevant Analysis is APPROXIMATE}
\end{frame}
}


\section{Extracting Meaning from Data}
\label{sec:extrct-meaning}

\subsection{Why is Data Analysis So Important?}
\begin{frame}
\frametitle{Gathering More Data}
\begin{itemize}
  \item States and districts collect hundreds of attributes about millions of students
  \item Data is collected before children reach school age and after they have moved to a college or a career
  \item Patterns can tell us how choices in policy will actually affect the population
  \item Lessons learned can help us build simulations to weigh policy outcomes before making decisions--decision support
\end{itemize}
\end{frame}

{
\usebackgroundtemplate{\includegraphics[width=\paperwidth]{datain}}
\begin{frame}[plain]
\frametitle{Data is like ore}
\end{frame}
}

{
\usebackgroundtemplate{\includegraphics[width=\paperwidth]{silvergoldbars}}
\begin{frame}[plain]
\frametitle{Analysis concentrates its value}
\end{frame}
}

{
\usebackgroundtemplate{\includegraphics[width=\paperwidth]{policyproduct}}
\begin{frame}[plain]
\frametitle{And it can be used to produce something}
\end{frame}
}



\subsection{Barriers to Data Analysis}

\begin{frame}
\frametitle{Institutional Frustrations}
We just need to get our jobs done. We need to do them efficiently, but also transparently and in a reproducible manner. This is currently costly in time, money, and management resources.
\end{frame}

{
\usebackgroundtemplate{\includegraphics[width=\paperwidth]{frustration}}
\begin{frame}[plain]
\end{frame}
}

\begin{frame}
\frametitle{Institutional Frustrations}
We just need to get our jobs done. We need to do them efficiently, but also transparently and in a reproducible manner. This is currently costly in time, money, and management resources.
  \begin{itemize}
  \item Acquiring proprietary tools from vendors takes agreements, legal documents, and lag time
  \pause
  \item Sharing data with external researchers requires legal agreements, levels of management approval, and planning time to specify narrow scope
  \end {itemize}
\end{frame}

{
\usebackgroundtemplate{\includegraphics[width=\paperwidth]{paperwork}}
\begin{frame}[plain]
\end{frame}
}

\begin{frame}
\frametitle{Institutional Frustrations}
  \begin{itemize}
  \item Analyses are often done in proprietary tool sets, poorly documented, and unable to be reproduced with updated data later
  \end {itemize}
\end{frame}

{
\usebackgroundtemplate{\includegraphics[width=\paperwidth]{jail}}
\begin{frame}[plain]
\end{frame}
}

\begin{frame}
\frametitle{What about analyses we do?}
\begin{center}
\Large Analyses we do complete are not used enough to drive decisions.
\end {center}
\end{frame}

{
\usebackgroundtemplate{\includegraphics[width=\paperwidth]{puzzleorg}}
\begin{frame}[plain]
\frametitle{Incoherence}
\end{frame}
}

\begin{frame}
\frametitle{Analyses Don't Get Used}
Often when we do an in house analysis it does not get used or only gets used once.
  \begin{itemize}
  \item In house analysis often relies on the expertise of one or two staff who are obligated elsewhere. 
  \pause
  \item Analysis are done using ad-hoc tools distributed among expertise of individual staff with no comprehensive standard. 
  \pause
  \item The information we have is dependent on individual staff and the analysis projects they undertake and their tenure supporting these efforts.
  \pause
  \item Staff turnover threatens to change the information available to make decisions as knowledge leaves the agency, breaking continuity with previous reports
  \pause
  \end{itemize}
\end{frame}

\begin{frame}
\frametitle{Or we do the wrong analysis}
\begin{center}
\Huge \textbf{Sometimes analyses are done at the whim of an analyst or two and not tied to the needs of decision makers or stakeholders}
\end{center}
\end{frame}

{
\usebackgroundtemplate{\includegraphics[width=\paperwidth]{irrelevant}}
\begin{frame}[plain]
\frametitle{Irrelevant}
\end{frame}
}

\begin{frame}
\frametitle{Example}
\begin{center}
\textbf{The state chief school officer asks: \\ 
``Do our state bilingual-bicultural programs provide any benefit to our students? Should we focus on ESL more or keep our BLBC programs?''}
\end{center}
\vspace{.1in}
Instead we answer:
\begin{itemize}
  \item Our ELL students are doing better than last year.
  \item National research is inconclusive on these programs.
  \item Our data shows participation is up in BLBC programs.
  \item We found a researcher to help with this in six months.
\end{itemize}
\end{frame}


\subsection{What is the solution?}

{
\usebackgroundtemplate{\includegraphics[width=\paperwidth]{openfield}}
\begin{frame}[plain]
\frametitle{Open Source Tools}
\end{frame}
}

{
\usebackgroundtemplate{\includegraphics[width=\paperwidth]{rscreenshot}}
\begin{frame}[plain]
\frametitle{R}
\end{frame}
}


\begin{frame}
\frametitle{R As the Solution}
  \begin{columns}
  \column[t]{.5\textwidth}
  Objections to Data Analysis
  \begin{itemize}
  \item Costly
  \item Slow and Time Consuming
  \item Technical and complex
  \item Opaque and not actionable
  \end{itemize}
  \column[t]{.5\textwidth}
  The R Solution
  \begin{itemize}
  \item R is free and open source
  \item R allows reproducible and sharable analysis across researchers
  \item R can be scripted to do common tasks
  \item R is a lingua franca that standardizes common tasks
  \end{itemize}
  \end{columns}
\end{frame}


\begin{frame}
\frametitle{Caveats}
\begin{center}
\Large But wait...? Isn't R?
\end{center}
\end{frame}

% But wait, isn't R difficult to use?
% Isn't open source software buggy?
% 

{
\usebackgroundtemplate{\includegraphics[width=\paperwidth]{maze}}
\begin{frame}[plain]
\frametitle{Confusing?}
\end{frame}
}

{
\usebackgroundtemplate{\includegraphics[width=\paperwidth]{buggy}}
\begin{frame}[plain]
\frametitle{Full of Bugs?}
\end{frame}
}

{
\usebackgroundtemplate{\includegraphics[width=\paperwidth]{hummer}}
\begin{frame}[plain]
\frametitle{Inefficient?}
\end{frame}
}


\begin{frame}
\frametitle{The Truth}
\begin{itemize}
   \item The short answer is no. 
   \pause
   \item R has a high startup cost, but we are working to bring that down. 
   \pause
   \item R has some quirks, but all software does.
   \pause
   \item And, R can be amazingly more efficient through collaboration and sharing of code and tools.
\end{itemize}
\end{frame}


\section{Introduction to R}
\label{sec:intro-r}

\subsection{What is R?}
\label{sec:what-r}


\begin{frame}
  \begin{itemize}
  \item R is an Open Source (and freely available) environment for statistical computinga nd graphics
  \item Available for Windows, Mac OS X, and Linux
  \item R is being actively developed with two major releases per year and dozens of releases of add on packages
  \item R can be extended with 'packages' that contain data, code, and documentation to add new functionality
  \end{itemize}
\end{frame}

\begin{frame}
\frametitle{More Support for R}
  \begin{itemize}
  \item R is a common tool among data experts at major universities
  \item No need to go through procurement, R can be installed in any environment on any machine and used with no licensing or agreements needed
  \item R source code is very readable to increase transparency of analyses
  \item R code is easily borrowed from and shared with others
  \item R is incredibly flexible and can be adapted to specific local needs
  \item R is under active development, improving greatly, and supported wildly by both professional and academic developers
  \end{itemize}
\end{frame}



\begin{frame}
\frametitle{Using R}
  \begin{columns}
  \column[t]{.5\textwidth}
  \begin{itemize}
  \item R can be used with an excellent Integrated Development Environment 
  \item RStudio makes many of the basic tasks in R much easier like
    \begin{enumerate}
    \item Importing data
    \item Previewing plots
    \item Version control 
    \item Collaboration
    \end{enumerate}
  \item Greatly increases ease of use
  \end{itemize}
\column[t]{.5\textwidth}
  \begin{center}
  \begin{figure}
  \includegraphics{rstudio-windows.png}
  \end{figure}
  \end{center}
\end{columns}
\end{frame}


\begin{frame}
\frametitle{Pros and Cons of R}
  \begin{columns}
  \column[t]{.5\textwidth}
  Pros of \textbf{R}
  \begin{itemize}
  \item Open source and freely available on all platforms
  \item Scripting for reproducible and transparent analyses
  \item Extensible to fit skills, needs, and cutting edge techniques
  \item Excellent graphical and output capabilities
  \end{itemize}
  \column[t]{.5\textwidth}
  Cons
  \begin{itemize}
  \item Steep learning curve and command line interface
  \item Requires specific inputs to get desired results
  \item Unforgiving of misspecification of inputs
  \item Data input can be tricky at first
  \end{itemize}
  \end{columns}
\end{frame}

\begin{frame}
\frametitle{How does it help?}
\begin{itemize}
  \item R can be built to do what you want without waiting for a vendor to upgrade software
  \pause
  \item All add-ons and functionality to R are free and shared immediately
  \pause 
  \item Student Growth Percentiles (SGP) package is available for growth modeling now
  \pause 
  \item It standardizes processes and makes them more transparent
\end{itemize}
\end{frame}

\begin{frame}
\frametitle{Standardization is Good}
\begin{itemize}
  \item By standardizing common data tasks staff are freed up to do other tasks
  \pause
  \item Wisconsin is using R to calculate AYP directly from LDS dataset
  \pause
  \item One script, one run, all reports and error checks run--saving weeks of work
  \pause
  \item Easy to understand how calculation is done protecting against discontinuity if staff turnsover
  \pause
  \item Other reports and data analyses are being standardized--quality checks on LDS data, etc.
\end{itemize}
\end{frame}
  


\subsection{What can R do?}

\begin{frame}
\frametitle{Visualization}
\vspace{.2in}

\large One of the major strengths of R is its ability to create informative and compelling visualizations of data.

\end{frame}

\begin{frame}
\frametitle{Examples of R Figures}
  \begin{figure}
  \begin{center}
\includegraphics[width=.3\textwidth]{hex} 
\includegraphics[width=.3\textwidth]{crosstab} \\
\includegraphics[width=.3\textwidth]{condensity}  
\includegraphics[width=.3\textwidth]{heatmap}
  \end{center}
  \end{figure}
\end{frame}

\subsection{R Examples}
\label{sec: rexamp}

\begin{frame}
\frametitle{Inference Trees}
\begin{center}
\vspace{-.1in}
\includegraphics[width=.8\textwidth]{classtree}
\end{center}
\end{frame}

\begin{frame}[containsverbatim]
\frametitle{Easy Code for Plot}
Code to make this plot:
\vspace{.1in}
\begin{Schunk}
\begin{Sinput}
> z1<-ctree(readSS~black+hisp+asian+indian+white+ell+disab+econ
+           +attday,data=subset(student_long,year=='2000',grade=5),
+           controls=bonf)
> plot(z1,type='simple',main="Splitting Categorical Data")
\end{Sinput}
\end{Schunk}
\end{frame}


\begin{frame}
\frametitle{Visual Crosstabs}
\vspace{.1in}
\includegraphics[width=.85\textwidth]{crosstab2}
\end{frame}

\begin{frame}[containsverbatim]
\frametitle{R Code}
Code for this plot that uses `mosaictabs' function from the LDS\_TOOLS package
\vspace{.1in}
\begin{Schunk}
\begin{Sinput}
> plotsub<-subset(student_long,year=='2001' & grade==6)
> varnames<-c('Female','Proficiency')
> mosaictabs.label(plotsub,plotsub$female,plotsub$proflvl,
+                 varnames,'Gender and Proficiency','Dummy Data 2012')
\end{Sinput}
\end{Schunk}
\end{frame}


\begin{frame}
\frametitle{Maps}
\begin{center}
\scalebox{.75}{
\includegraphics[width=.85\textwidth]{FRL2010}
}
\end{center}
\end{frame}

\begin{frame}
\frametitle{Heatmap}
\begin{center}
\scalebox{.75}{
\includegraphics[width=.85\textwidth]{heatmap}
}
\end{center}
\end{frame}

\begin{frame}
\frametitle{Conditional Density}
\begin{center}
\scalebox{.75}{
\includegraphics[width=.85\textwidth]{condensity}
}
\end{center}
\end{frame}

\begin{frame}
\frametitle{Faceted Hex}
\vspace{-.1in}
\begin{center}
\scalebox{.75}{
\includegraphics[width=.85\textwidth]{hex}
}
\end{center}
\end{frame}


\begin{frame}
\frametitle{Polished Scatter}
\begin{center}
\scalebox{.9}{
\includegraphics[width=.85\textwidth]{occupationscatter}
}
\end{center}
\end{frame}


\begin{frame}
\frametitle{Longitudinal Data}
\vspace{-.1in}
\begin{center}
\scalebox{.85}{
\includegraphics[width=.95\textwidth]{g3belowproficientreadingscores}
}
\end{center}
\end{frame}


\begin{frame}
\frametitle{Proficiency Polygon}
\vspace{-.13in}
\begin{center}
\scalebox{.82}{
\includegraphics[width=.95\textwidth]{proficiencypolygon}
}
\end{center}
\end{frame}



\begin{frame}
\frametitle{Individual Growth Trajectories}
\vspace{-.2in}
\begin{center}
\scalebox{.85}{
\includegraphics[width=.95\textwidth]{facetstudents}
}
\end{center}
\end{frame}


% Bug in Sweave compiling this chart in Sweave environment
% Try to reproduce and report bugfix to ggplot2 and Sweave teams




\begin{frame}
\frametitle{Individual Growth Trajectories}
\vspace{-.1in}
\begin{center}
\scalebox{.8}{
\includegraphics[width=.9\textwidth]{facetstudentspoly}
}
\end{center}
\end{frame}

\begin{Schunk}
\begin{Soutput}
null device 
          1 
\end{Soutput}
\end{Schunk}

\begin{frame}
\frametitle{Communicate Statistical Models}
\vspace{.05in}
\begin{center}
\scalebox{.9}{
\includegraphics[width=.9\textwidth]{coefplot1}
}
\end{center}
\end{frame}




\begin{frame}
\frametitle{Communicate Statistical Models II }
\vspace{.05in}
\begin{center}
\scalebox{.9}{
\includegraphics[width=.9\textwidth]{effectsize}
}
\end{center}
\end{frame}


\subsection{Getting StaRted}
\label{sec: start}

\begin{frame}[containsverbatim]
\frametitle{The Command Line}
\begin{itemize}
\item R can be tricky because it uses command lines.
\item This is powerful, but requires a learning curve.
\item Some simple calculations can give a feel for how R works
\end{itemize}
\begin{Schunk}
\begin{Sinput}
> 2+2
\end{Sinput}
\begin{Soutput}
[1] 4
\end{Soutput}
\begin{Sinput}
> 7*4
\end{Sinput}
\begin{Soutput}
[1] 28
\end{Soutput}
\begin{Sinput}
> exp(3)
\end{Sinput}
\begin{Soutput}
[1] 20.08554
\end{Soutput}
\begin{Sinput}
> pi
\end{Sinput}
\begin{Soutput}
[1] 3.141593
\end{Soutput}
\end{Schunk}
\end{frame}

\begin{frame}[containsverbatim]
\frametitle{Deconstruct R Commands}
\begin{Schunk}
\begin{Sinput}
> summary(student_long[,28:30])
\end{Sinput}
\begin{Soutput}
     readSS          mathSS     
 Min.   :200.0   Min.   :200.0  
 1st Qu.:430.9   1st Qu.:420.8  
 Median :494.4   Median :481.2  
 Mean   :494.9   Mean   :483.7  
 3rd Qu.:558.4   3rd Qu.:543.8  
 Max.   :850.4   Max.   :857.5  
        proflvl      
 below basic: 35927  
 basic      : 85983  
 proficient :198393  
 advanced   :129697  
\end{Soutput}
\end{Schunk}
\begin{itemize}
  \item \textbf{summary} is the function
  \item \textbf{student\_ long} is the data object
  %\item \textbf{[,27:31]} is an operator that specified conditions to operate under
\end{itemize}
\end{frame}

\begin{frame}[containsverbatim]
\frametitle{Simple R Operations}
\begin{Schunk}
\begin{Sinput}
> with(student_long,mean(readSS[year=='2001' & grade==4]))
\end{Sinput}
\begin{Soutput}
[1] 444.7696
\end{Soutput}
\end{Schunk}
\begin{Schunk}
\begin{Sinput}
> with(student_long,median(readSS[year=='2001' & grade==4]))
\end{Sinput}
\begin{Soutput}
[1] 442.1458
\end{Soutput}
\end{Schunk}
\begin{Schunk}
\begin{Sinput}
> with(student_long,max(readSS[year=='2001' & grade==4]))
\end{Sinput}
\begin{Soutput}
[1] 719.052
\end{Soutput}
\end{Schunk}
\begin{Schunk}
\begin{Sinput}
> with(student_long,min(readSS[year=='2001' & grade==4]))
\end{Sinput}
\begin{Soutput}
[1] 200
\end{Soutput}
\end{Schunk}
\begin{Schunk}
\begin{Sinput}
> with(student_long,summary(readSS[year=='2001' & grade==4]))
\end{Sinput}
\begin{Soutput}
   Min. 1st Qu.  Median    Mean 3rd Qu.    Max. 
  200.0   393.3   442.1   444.8   495.2   719.1 
\end{Soutput}
\end{Schunk}
\end{frame}

  
\begin{frame}[containsverbatim]
\frametitle{Crosstabs}
Let's test for balance among some categories of students 
\begin{Schunk}
\begin{Sinput}
> with(subset(student_long,year=='2001' 
+             & grade==3),table(female,race))
\end{Sinput}
\begin{Soutput}
      race
female    W    B    H    I    A
     0 5570 5081 1365  107  224
     1 5653 5252 1394  116  238
\end{Soutput}
\end{Schunk}
\begin{Schunk}
\begin{Sinput}
> #As proportions
> with(subset(student_long,year=='2001' 
+             & grade==3),round(prop.table
+             (table(female,race))*100),4)
\end{Sinput}
\begin{Soutput}
      race
female  W  B  H  I  A
     0 22 20  5  0  1
     1 23 21  6  0  1
\end{Soutput}
\end{Schunk}
\end{frame}

\begin{frame}
\frametitle{Crosstabs}
We can even output the results of R commands into a print-ready format. As we have below.
% latex table generated in R 2.14.1 by xtable 1.6-0 package
% Mon Feb 13 13:21:06 2012
\begin{table}[ht]
\begin{center}
\begin{tabular}{rrrrrr}
  \hline
 & W & B & H & I & A \\ 
  \hline
0 & 22.00 & 20.00 & 5.00 & 0.00 & 1.00 \\ 
  1 & 23.00 & 21.00 & 6.00 & 0.00 & 1.00 \\ 
   \hline
\end{tabular}
\end{center}
\end{table}\end{frame}

\section{More Advanced Functions}
\subsection{Analysis}

\begin{frame}
\frametitle{Doing More than the Basics}
  \begin{itemize}
  \item R can routinize basic functions like tables, crosstabs, and visualization of data
  \item R can also be extended to do more advanced analyses like multilevel modeling, spatial error modeling, Bayesian data analysis, forecasting, and simulation
  \item R can do advanced graphical functions as well
  \item R can even be expanded to incorporate additional programming languages like Python, C++, and Java
  \end{itemize}
\textbf{The downside of this is that these functions can have a steep learning curve.}
\end{frame}

\begin{frame}[containsverbatim]
\frametitle{ANOVA}
We can also do statistical tests using both Bayesian and Frequentist methods.
\begin{Schunk}
\begin{Sinput}
> nova1<-aov(readSS~female*race*econ,data=novaset)
> summary(nova1)
\end{Sinput}
\begin{Soutput}
                    Df   Sum Sq  Mean Sq  F value
female               1  2176464  2176464  766.610
race                 4 33929701  8482425 2987.743
econ                 1 20785160 20785160 7321.103
female:race          4    50009    12502    4.404
female:econ          1    12835    12835    4.521
race:econ            4   266621    66655   23.478
female:race:econ     4    18117     4529    1.595
Residuals        24980 70920091     2839         
                  Pr(>F)    
female           < 2e-16 ***
race             < 2e-16 ***
econ             < 2e-16 ***
female:race      0.00147 ** 
female:econ      0.03349 *  
race:econ        < 2e-16 ***
female:race:econ 0.17246    
Residuals                   
---
Signif. codes:  0 '***' 0.001 '**' 0.01 '*' 0.05 '.' 0.1 ' ' 1 
\end{Soutput}
\end{Schunk}
\end{frame}

\begin{frame}
\frametitle{Pretty Output}
We can also do print-ready model outputs with R's extensible formatting
% latex table generated in R 2.14.1 by xtable 1.6-0 package
% Mon Feb 13 13:21:07 2012
\begin{table}[ht]
\begin{center}
\begin{tabular}{lrrrrr}
  \hline
 & Df & Sum Sq & Mean Sq & F value & Pr($>$F) \\ 
  \hline
female           & 1 & 2176463.53 & 2176463.53 & 766.61 & 0.0000 \\ 
  race             & 4 & 33929701.29 & 8482425.32 & 2987.74 & 0.0000 \\ 
  econ             & 1 & 20785160.43 & 20785160.43 & 7321.10 & 0.0000 \\ 
  female:race      & 4 & 50008.56 & 12502.14 & 4.40 & 0.0015 \\ 
  female:econ      & 1 & 12835.21 & 12835.21 & 4.52 & 0.0335 \\ 
  race:econ        & 4 & 266620.58 & 66655.15 & 23.48 & 0.0000 \\ 
  female:race:econ & 4 & 18116.94 & 4529.24 & 1.60 & 0.1725 \\ 
  Residuals        & 24980 & 70920091.02 & 2839.07 &  &  \\ 
   \hline
\end{tabular}
\end{center}
\end{table}\end{frame}

\subsection{Linear Model Example}

\begin{frame}[containsverbatim,allowframebreaks,fragile]
\frametitle{A simple OLS Model}
\begin{Schunk}
\begin{Sinput}
> mod1<-lm(readSS~female*race*econ+grade*year,data=student_long)
> summary(mod1)
\end{Sinput}
\begin{Soutput}
Call:
lm(formula = readSS ~ female * race * econ + grade * year, data = student_long)

Residuals:
     Min       1Q   Median       3Q      Max 
-284.566  -34.219    0.104   34.321  246.441 

Coefficients:
                    Estimate Std. Error  t value
(Intercept)        383.11107    0.49828  768.865
female              14.24756    0.32404   43.969
raceB              -37.21069    0.57997  -64.160
raceH              -41.01409    0.78264  -52.405
raceI                2.48529    2.15941    1.151
raceA                0.45644    1.38578    0.329
econ1              -54.71664    0.32522 -168.246
grade               20.81004    0.07791  267.111
year2001            -3.48643    0.63452   -5.495
year2002           134.55881    0.63452  212.065
female:raceB        -0.88573    0.82287   -1.076
female:raceH         4.46915    1.09121    4.096
female:raceI         4.72824    2.88579    1.638
female:raceA        -0.04431    1.93005   -0.023
female:econ1        -3.22112    0.45766   -7.038
raceB:econ1        -11.97227    0.64901  -18.447
raceH:econ1          6.46426    0.89487    7.224
raceI:econ1        -25.56913    2.62396   -9.744
raceA:econ1        -23.59874    1.72464  -13.683
grade:year2001       9.50620    0.11018   86.281
grade:year2002      -3.95024    0.11018  -35.853
female:raceB:econ1  -0.29569    0.91935   -0.322
female:raceH:econ1  -2.03755    1.25022   -1.630
female:raceI:econ1   1.70605    3.57366    0.477
female:raceA:econ1  -2.64903    2.39991   -1.104
                   Pr(>|t|)    
(Intercept)         < 2e-16 ***
female              < 2e-16 ***
raceB               < 2e-16 ***
raceH               < 2e-16 ***
raceI                 0.250    
raceA                 0.742    
econ1               < 2e-16 ***
grade               < 2e-16 ***
year2001           3.92e-08 ***
year2002            < 2e-16 ***
female:raceB          0.282    
female:raceH       4.21e-05 ***
female:raceI          0.101    
female:raceA          0.982    
female:econ1       1.95e-12 ***
raceB:econ1         < 2e-16 ***
raceH:econ1        5.07e-13 ***
raceI:econ1         < 2e-16 ***
raceA:econ1         < 2e-16 ***
grade:year2001      < 2e-16 ***
grade:year2002      < 2e-16 ***
female:raceB:econ1    0.748    
female:raceH:econ1    0.103    
female:raceI:econ1    0.633    
female:raceA:econ1    0.270    
---
Signif. codes:  0 '***' 0.001 '**' 0.01 '*' 0.05 '.' 0.1 ' ' 1 

Residual standard error: 51.53 on 449975 degrees of freedom
Multiple R-squared: 0.6742,	Adjusted R-squared: 0.6742 
F-statistic: 3.88e+04 on 24 and 449975 DF,  p-value: < 2.2e-16 
\end{Soutput}
\end{Schunk}
\end{frame}

\begin{frame}
\frametitle{Model Evaluation}
\begin{center}
\vspace{-.1in}
\includegraphics[width=.8\textwidth]{residplot}
\end{center}
\end{frame}

\begin{frame}
\frametitle{Model Evaluation Part 2}
\vspace{-.1in}
\begin{center}
\includegraphics[width=.8\textwidth]{fittedplot}
\end{center}
\end{frame}

\begin{frame}
\frametitle{Better Fitting}
Using advanced techniques we can greatly improve our model fit over the OLS model.

\vspace{-.1in}
\begin{center}
\includegraphics[height=.65\paperheight]{mlfitted}
\end{center}
\end{frame}


\begin{frame}
\frametitle{Compare OLS and Mixed Effects}
A simple mixed-effects model estimated with an R package can outperform the simple OLS without much additional effort
\vspace{-.1in}
\begin{center}
\includegraphics[height=.6\paperheight]{OLSandMLM}
\end{center}
\end{frame}


\section{Putting It Together to Collaborate}

\begin{frame}
\frametitle{Some stuff}
The advantage of R comes when we specialize to develop \textcolor{red}{advanced tools} that are compatible across our \textcolor{blue}{very similar datasets.} 
\end{frame}


{
\usebackgroundtemplate{\includegraphics[width=\paperwidth]{hackathon}}
\begin{frame}[plain]
\frametitle{Collaborative Development}
\end{frame}
}


\subsection{Same Data, Similar Analyses}

\begin{frame}
\frametitle{Leverage Similar Data}
\Large Why can we do this?
\begin{itemize}
  \item We all share similar data with similar attributes and similar reporting needs
  \item Standardizing on our analysis language makes applying analysis from one place to another easy due to data similarities
  \item \Large \textbf{\textcolor{green}{Build it.} \textcolor{blue}{Share it.} \textcolor{red}{Use it.}}
\end{itemize}
\end{frame}



\subsection{Coordinating and Social Coding}
\begin{frame}
\frametitle{GitHub}
\Large \textbf{\textcolor{blue}{GitHub}} \normalsize provides an excllent way to do this. 
\vspace{.5in}
\begin{center}
\includegraphics[width=.45\textwidth]{GitHublogo}
\end{center}
\end{frame}



\begin{frame}
\frametitle{LDS\_TOOLS}
\vspace{.18in}
\Large  \textcolor{blue}{\href{https://github.com/jknowles/LDS_TOOLS}{https://github.com/jknowles/LDS\_TOOLS}}
\begin{itemize}
  \item Uses the \textbf{`git'} version control system to track collaborative coding on the same source document
  \item Free and open source coding environment that plays well with RStudio
  \item No need to contribute, provides easy way to access work of others
\end{itemize}
\end{frame}

\begin{frame}
\frametitle{Open Source Analysis Code}
\begin{itemize}
  \item LDS\_TOOLS is a fledgling effort to open source many of the graphics and analyses from earlier in this presentation
  \item Make R code and \LaTeX{} code available to be applied to other SEA and LEA data
  \item Packaged with a dummy dataset representing common educational data attributes for testing and sharing 
  \item Share visualization techniques, statistical models, and even full reports
  \item Plug and play--change the variables to match your data and produce the same visualization, report, analysis
\end{itemize}
\end{frame}

\begin{frame}
\frametitle{LDS\_TOOLS Available Now}
You can visit online at: \textcolor{blue}{\href{https://github.com/jknowles/LDS_TOOLS}{https://github.com/jknowles/LDS\_TOOLS}}
\vspace{.1in}
\begin{columns}
  \column[t]{.5\textwidth}
  \textbf{Available Tools:}
    \begin{itemize}
      \item Improved mosaic plots
      \item Gantt charts for project planning
      \item Proficiency polygons for assessments
      \item Convenience functions for education data
    \end{itemize}
  \column[t]{.5\textwidth}
  \textbf{Planned Tools:}
    \begin{itemize}
      \item Easy to make maps
      \item Data mining and statistical modeling routines
      \item Pre-built data quality reports
      \item Summary reports for NSC, Assessment, and Enrollment data
    \end{itemize}
\end{columns}
\end{frame}


\begin{frame}
\frametitle{Who can get involved?}
\Large Current staff can use these tools, and even contribute to them
\normalsize
\begin{itemize}
  \item Analysts with a couple of days to learn basic R skills
  \item Anyone with programming skills
  \item University partners
  \item The R Community
\end{itemize}
\end{frame}



\begin{frame}
\frametitle{Example}
\vspace{.6in}
\large This entire presentation was created with R, \LaTeX{} and is available online at GitHub to be edited, modified, and reused. 
\end{frame}

\begin{frame}
\frametitle{Tools}

\begin{itemize}
  \item \textbf{R} (\href{http://cran.r-project.org/}{http://cran.r-project.org/})
    \begin{itemize}
    \item An open source statistics package that is freely available for all platforms.
    \end{itemize}
  \item \textbf{RStudio} (\href{http://www.rstudio.org/}{http://www.rstudio.org/})
    \begin{itemize}
    \item An enhanced front-end for R. An Integrated Development Environment (\textsc{IDE}) for statistical programming.
    \end{itemize}
  \item \textbf{Quantum GIS} (\href{http://www.qgis.org/}{http://www.qgis.org/})
    \begin{itemize}
    \item A GIS package that provides most of the functionality of ArcGIS but is freely available.
    \end{itemize}
  \item \textbf{GeoDa} (\href{http://geodacenter.asu.edu/}{http://geodacenter.asu.edu/})
    \begin{itemize}
    \item A geo-spatial statistics package for analyzing clustering and spatial correlation of datasets.
    \end{itemize}
  \item \textbf{\LaTeX{}} (\href{http://www.latex-project.org/}{http://www.latex-project.org/})
    \begin{itemize}
    \item A typesetting and document building tool that integrates with R.
    \end{itemize}
  \end{itemize}
\end{frame}

\begin{frame}
\frametitle{Tutorials}
\begin{itemize}
  \item R Reference (\href{http://www.statmethods.net/}{http://www.statmethods.net/})
  \item First R Commands to Learn (\href{https://github.com/hadley/devtools/wiki/vocabulary}{https://github.com/hadley/devtools/wiki/vocabulary})
  \item Beginning with \LaTeX{} (\href{http://en.wikibooks.org/wiki/LaTeX}{http://en.wikibooks.org/wiki/LaTeX})
  \item Quantum GIS Guide (\href{http://qgis.org/en/documentation/manuals.html}{http://qgis.org/en/documentation/manuals.html})
  \item R Graph Gallery (\href{http://addictedtor.free.fr/graphiques/}{http://addictedtor.free.fr/graphiques/})
\end{itemize}
\end{frame}



\end{document}
