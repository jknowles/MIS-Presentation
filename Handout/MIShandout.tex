\documentclass{article}
\usepackage{dcolumn, multirow}
\usepackage{epsfig,graphicx}
\usepackage{verbatim, rotating, paralist,hyperref}
\usepackage{float}
\graphicspath{{../img/}} %additional graphics directory
\usepackage[margin=0.75in]{geometry}

\begin{document}
\setcounter{page}{0}
\thispagestyle{empty}
\begin{center}

\Large \textbf{Policy Relevant Visualization and Analysis of LDS Data with Open Source Tools} \\
\vspace{.15in}

\normalsize \textbf{Jared Knowles, Policy Research Advisor} \\
\vspace{6pt}
\normalsize \textbf{Wisconsin Department of Public Instruction} \\
\vspace{6pt}
\scriptsize $\bullet$ jared.knowles@dpi.wi.gov $\bullet$ \href{http://github.com/jknowles/}{http://github.com/jknowles/} $\bullet$ \today \\
\end{center}

\line(1,0){500}

\vspace{.05in}

\textbf{Open Source Tools}

\begin{itemize}
  \item \textbf{R} (\href{http://cran.r-project.org/}{http://cran.r-project.org/})
    \begin{itemize}
    \item An open source statistics package that is freely available for all platforms.
    \end{itemize}
  \item \textbf{RStudio} (\href{http://www.rstudio.org/}{http://www.rstudio.org/})
    \begin{itemize}
    \item An enhanced front-end for R. An Integrated Development Environment (\textsc{IDE}) for statistical programming.
    \end{itemize}
  \item \textbf{Quantum GIS} (\href{http://www.qgis.org/}{http://www.qgis.org/})
    \begin{itemize}
    \item A GIS package that provides most of the functionality of ArcGIS but is freely available.
    \end{itemize}
%  \item \textbf{GeoDa} (\href{http://geodacenter.asu.edu/}{http://geodacenter.asu.edu/})
%    \begin{itemize}
%    \item A geo-spatial statistics package for analyzing clustering and spatial correlation of datasets.
%    \end{itemize}
  \item \textbf{\LaTeX{}} (\href{http://www.latex-project.org/}{http://www.latex-project.org/})
    \begin{itemize}
    \item A typesetting and document building tool that integrates with R.
    \end{itemize}
  \item \textbf{git} 
  (\href{http://git-scm.com/}{http://git-scm.com/})
    \begin{itemize}
    \item A version control system for collaborative coding that works with R.
    \end{itemize}    

\end{itemize}

\vspace{.15in}

\textbf{Tutorials and Help Getting Started}

\begin{itemize}
  \item R Reference (\href{http://www.statmethods.net/}{http://www.statmethods.net/})
  \item First R Commands to Learn (\href{https://github.com/hadley/devtools/wiki/vocabulary}{https://github.com/hadley/devtools/wiki/vocabulary})
  \item Beginning with \LaTeX{} (\href{http://en.wikibooks.org/wiki/LaTeX}{http://en.wikibooks.org/wiki/LaTeX})
  \item Quantum GIS Guide (\href{http://qgis.org/en/documentation/manuals.html}{http://qgis.org/en/documentation/manuals.html})
  \item R Graph Gallery (\href{http://addictedtor.free.fr/graphiques/}{http://addictedtor.free.fr/graphiques/})

\end{itemize}

\vspace{.15in}

\textbf{Collaboration on LDS\_TOOLS}

\begin{itemize}
  \item \textsc{LDS\_TOOLS} Package for R (\href{https://github.com/jknowles/LDS_TOOLS}{https://github.com/jknowles/LDS\_TOOLS})
    \begin{itemize}
    \item  A project for R that seeks to make it easier for administrators at state and local education agencies to analyze and visualize their data on student, school, and district performance.
    \item  The project is open source and available for anyone to contribute to, modify, download, copy, and/or share. 
    \item  Interested folks with programming skills especially at the \textsc{SEA}, districts, or \textsc{REL}s should visit get involved.
    \item  Currently it comes with simulated achievement data to allow commands to be tested on data that closely resembles administrative records common across state and district data systems.
    \item GitHub will be used to coordinate these efforts. 
    \end{itemize}

\end{itemize}

\noindent You can download this handout, the presentation slides, all images from the presentation and more at the GitHub repository for this presentation: \href{https://github.com/jknowles/mis-presentation}{https://github.com/jknowles/mis-presentation}. Just click "Download ZIP".
\vspace{.2in}

\line(1,0){500}

\vfill{}



\begin{center}
\begin{tabular}{|c|c|c|}
\includegraphics[height=.05\paperheight]{dpilogo} &
\includegraphics[height=.05\paperheight]{GitHublogo}&
\includegraphics[height=.05\paperheight]{Rlogo} 
\end{tabular}
\end{center}




\end{document}
